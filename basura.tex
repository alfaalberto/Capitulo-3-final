\documentclass[10pt]{beamer}

%------------------------------------------------------------
% Configuraciones generales
%------------------------------------------------------------
\usepackage[utf8]{inputenc}
\usepackage[spanish]{babel} % Para textos automáticos en español
\usepackage{amsmath, amssymb} % Paquetes matemáticos
\usepackage{graphicx} % Para incluir imágenes
\usepackage{tikz} % Para dibujar gráficos
\usetikzlibrary{arrows.meta, positioning, calc, matrix, fit, backgrounds, shapes.multipart}
\usepackage{array} % Para mejorar tablas
\usepackage{multicol} % Para usar múltiples columnas fuera del entorno columns de beamer
% \usepackage{caption} % ELIMINADO - Beamer maneja captions. Puede causar conflictos.

%------------------------------------------------------------
% Tema y fuentes
%------------------------------------------------------------
\usetheme{Madrid} % Tema sobrio y profesional
\usecolortheme{default}
\usefonttheme{professionalfonts} % Fuentes profesionales

% Ajuste de color para el cuerpo de los exampleblock para asegurar legibilidad
\setbeamercolor{block body example}{bg=green!10!white, fg=black}
% El tema Madrid ya define block title example con un verde oscuro y texto blanco,
% pero si se quisiera forzar sería algo como:
% \setbeamercolor{block title example}{bg=green!60!black, fg=white}


%------------------------------------------------------------
% Información de la presentación
%------------------------------------------------------------
\title[Filtrado Espacial]{Fundamentos del Filtrado Espacial}
\subtitle{Procesamiento Digital de Imágenes}
\author[Tu Nombre/Institución]{Nombre del Presentador \\ \textit{Institución/Departamento}}
\institute[] % Puedes añadir tu instituto aquí si es diferente

%------------------------------------------------------------
% Comandos personalizados
%------------------------------------------------------------
\newcommand{\matlabel}[1]{\text{\scriptsize #1}}
\newcommand{\eqnum}[1]{\tag{#1}} % Para numeración manual de ecuaciones como en el texto

%------------------------------------------------------------
% Inicio del documento
%------------------------------------------------------------
\begin{document}

%------------------------------------------------------------
% Diapositiva de título
%------------------------------------------------------------
\begin{frame}
\titlepage
\end{frame}

%------------------------------------------------------------
% Diapositiva de índice
%------------------------------------------------------------
\begin{frame}{Contenido}
\tableofcontents
\end{frame}

%------------------------------------------------------------
% Sección 1: Introducción al Filtrado Espacial
%------------------------------------------------------------
\section{Introducción al Filtrado Espacial}

\begin{frame}{Definición y Propósito}
\begin{block}{Filtrado Espacial}
El \textbf{filtrado espacial} se refiere a la modificación del valor de cada píxel de una imagen en función de los valores del propio píxel y sus vecinos.
\vspace{0.1cm}
Es una técnica fundamental en el procesamiento de imágenes, utilizada principalmente para \textbf{realce (enhancement)} y restauración.
\end{block}
\vspace{0.1cm}
El nombre "filtro" se toma del procesamiento en el \textbf{dominio de la frecuencia} (Capítulo 4 del material original), donde "filtrar" implica:
\begin{itemize}
\item Pasar, modificar o rechazar componentes de frecuencia específicos de una imagen.
\item Ejemplo: Un \textit{filtro paso-bajas} atenúa las altas frecuencias, resultando en un suavizado o difuminado (blurring) de la imagen.
\item Efectos similares de suavizado se pueden lograr directamente en el dominio espacial.
\end{itemize}
\end{frame}

\begin{frame}{Tipos de Filtros Espaciales}
El filtrado espacial modifica una imagen reemplazando el valor de cada píxel por una función de los valores del píxel y sus vecinos.
\vspace{0.3cm}
\begin{columns}[T]
\begin{column}{0.5\textwidth}
\begin{alertblock}{Filtro Lineal Espacial}
La operación realizada sobre los píxeles de la imagen es \textbf{lineal}.
\vspace{0.1cm}
Nos centraremos inicialmente en estos filtros.
\end{alertblock}
\end{column}
\begin{column}{0.5\textwidth}
\begin{exampleblock}{Filtro No Lineal Espacial}
La operación realizada sobre los píxeles es \textbf{no lineal}.
\vspace{0.1cm}
Se introducirán algunos filtros no lineales básicos más adelante (referencia a Sección 5.3 del material original).
\end{exampleblock}
\end{column}
\end{columns}
\end{frame}

%------------------------------------------------------------
% Sección 2: Mecánica del Filtrado Espacial Lineal
%------------------------------------------------------------
\section{Mecánica del Filtrado Lineal}

\begin{frame}{El Kernel del Filtro}
Un filtro espacial lineal realiza una operación de \textbf{suma de productos} entre una sub-área de la imagen $f$ y una matriz llamada \textbf{kernel de filtro} $w$.
\begin{itemize}
\item El \textbf{kernel} (o núcleo) es un arreglo (generalmente 2D) de coeficientes.
\item Sus \textbf{coeficientes} determinan la naturaleza y el comportamiento del filtro (e.g., suavizado, realce de bordes).
\item El \textbf{tamaño del kernel} define la vecindad de píxeles considerada para cada operación.
\item \textbf{Otros términos comunes} para el kernel:
{\footnotesize
\begin{itemize}
\item Máscara (mask)
\item Plantilla (template)
\item Ventana (window)
\end{itemize}
}
\end{itemize}
\end{frame}

\begin{frame}[fragile]{Operación con un Kernel $3 \times 3$}\tiny
\begin{block}{Respuesta del Filtro $g(x,y)$}
En cualquier punto $(x,y)$ de la imagen, la respuesta $g(x,y)$ del filtro es la suma de productos de los coeficientes del kernel y los píxeles de la imagen cubiertos por el kernel:
\begin{align}
g(x,y) = &w(-1,-1)f(x-1,y-1) + w(-1,0)f(x-1,y) + w(-1,1)f(x-1,y+1) \nonumber \\
&+ w(0,-1)f(x,y-1)   + w(0,0)f(x,y)     + w(0,1)f(x,y+1) \nonumber \\
&+ w(1,-1)f(x+1,y-1)   + w(1,0)f(x+1,y)     + w(1,1)f(x+1,y+1) \eqnum{3-30}
\end{align}
\end{block}
\vspace{0.1cm}
{\footnotesize
\begin{itemize}
\item A medida que las coordenadas $x$ e $y$ varían, el centro del kernel se mueve de píxel a píxel, generando la imagen filtrada $g$.
\item El coeficiente central del kernel, $w(0,0)$, se alinea con el píxel en la ubicación $(x,y)$ de la imagen original $f$.
\end{itemize}
}
\end{frame}

\begin{frame}[fragile]{Ilustración del Filtrado Lineal (Fig. 3.28)}
\begin{figure}
\centering
\begin{tikzpicture}[scale=0.6, every node/.style={scale=0.7}]
% Imagen f
\node[draw, minimum width=3.5cm, minimum height=3cm, label=above:{Imagen $f$}] (img_f) at (-2.5,0) {};
\node[font=\tiny, align=center] at (img_f.center) {Píxeles};
\draw[gray, thin] (img_f.north west) grid (img_f.south east);
\node[font=\tiny, below left] at (img_f.north west) {Origen img.};

% Kernel w
\node[draw, minimum size=1.2cm, label=above:{Kernel $w$}] (kernel_w) at (2.5,1.5) {};
\matrix (kernel_matrix) [matrix of math nodes, nodes={draw, minimum size=0.4cm, scale=0.6}, ampersand replacement=\&, at=(kernel_w.center)] {
w_{-1,-1} \& w_{-1,0} \& w_{-1,1} \\
w_{0,-1} \& w_{0,0} \& w_{0,1} \\
w_{1,-1} \& w_{1,0} \& w_{1,1} \\
};
\node[circle, fill=red, inner sep=0.5pt, label={[scale=0.6]right:Origen ker.}] at (kernel_matrix-2-2.center) {};

% Detalle de superposición
\begin{scope}[xshift=2.5cm, yshift=-1.5cm]
\node[draw, label=above:{\tiny Píxeles bajo kernel en $(x,y)$}] (pixel_grid_label) {};
\matrix (pixel_grid) [matrix of math nodes, nodes={draw, minimum size=0.6cm, scale=0.6}, ampersand replacement=\&, below=0.05cm of pixel_grid_label] {
f(x-1,y-1) \& f(x-1,y) \& f(x-1,y+1) \\
f(x,y-1)   \& f(x,y)   \& f(x,y+1) \\
f(x+1,y-1) \& f(x+1,y) \& f(x+1,y+1) \\
};
\node[font=\tiny, red, scale=0.6] at (pixel_grid-2-2) {$w(0,0)$ aquí};
\end{scope}

\draw[-latex, thick, blue] ($(img_f.east)+(0.1,0)$) to [bend left=10] node[midway, above, sloped, scale=0.6] {Deslizamiento} ($(kernel_w.west)-(0.1,0)$);
\end{tikzpicture}
\caption{\footnotesize Mecánica del filtrado espacial lineal (conceptual, Fig. 3.28). El kernel se desliza sobre la imagen, calculando la suma de productos en $(x,y)$.}
\end{figure}
\begin{alertblock}{\footnotesize Orígenes}
{\footnotesize El origen de la imagen suele ser la esquina superior izquierda. Para simplificar las expresiones, el origen del kernel se define en su centro, especialmente para kernels simétricos.}
\end{alertblock}
\end{frame}

\begin{frame}[fragile]{Expresión General y Tamaño del Kernel}
Para un kernel de tamaño $m \times n$, se asume que $m = 2a+1$ y $n = 2b+1$, donde $a, b$ son enteros no negativos. Esto implica que nos centramos en kernels de \textbf{tamaño impar} en ambas direcciones.
\vspace{0.1cm}
El filtrado espacial lineal de una imagen de tamaño $M \times N$ con un kernel de $m \times n$ está dado por:
\begin{equation}
g(x,y) = \sum_{s=-a}^{a} \sum_{t=-b}^{b} w(s,t)f(x+s,y+t) \eqnum{3-31}
\end{equation}
Donde $x$ e $y$ varían para que el centro (origen) del kernel visite cada píxel en $f$.
\vspace{0.1cm}
\begin{exampleblock}{\footnotesize Herramienta Central}
{\footnotesize La ecuación (3-31) es una herramienta central en el filtrado lineal. Para un valor fijo de $(x,y)$, implementa la suma de productos como en (3-30), pero para un kernel de tamaño impar arbitrario.}
\end{exampleblock}
\end{frame}

%------------------------------------------------------------
% Sección 3: Correlación y Convolución Espacial
%------------------------------------------------------------
\section{Correlación y Convolución}

\begin{frame}[fragile]{Definición de Correlación Espacial}
La \textbf{correlación espacial} es el proceso ilustrado gráficamente en la Fig. 3.28 y descrito matemáticamente por la Ec. (3-31).
{\footnotesize
\begin{itemize}
\item Consiste en mover el centro de un kernel sobre una imagen.
\item En cada ubicación, se calcula la suma de productos entre los coeficientes del kernel y los píxeles correspondientes de la imagen.
\end{itemize}
}
\vspace{0.1cm}
La correlación de un kernel $w$ con una imagen $f(x,y)$, denotada $(w \star f)(x,y)$, es:
\begin{equation}
(w \star f)(x,y) = \sum_{s=-a}^{a} \sum_{t=-b}^{b} w(s,t)f(x+s,y+t) \eqnum{3-34}
\end{equation}
\vspace{0.1cm}
Para el caso 1D, la Ec. (3-31) se convierte en:
\begin{equation}
g(x) = \sum_{s=-a}^{a} w(s)f(x+s) \eqnum{3-32}
\end{equation}
\end{frame}

\begin{frame}[fragile]{Definición de Convolución Espacial}
La mecánica de la \textbf{convolución espacial} es la misma que la correlación, con una diferencia clave:
{\footnotesize
\begin{itemize}
\item El kernel de correlación se \textbf{rota 180°} antes de realizar la operación de deslizamiento y suma de productos.
\end{itemize}
}
\vspace{0.1cm}
La convolución de un kernel $w$ de tamaño $m \times n$ con una imagen $f(x,y)$, denotada $(w * f)(x,y)$, se define como:
\begin{equation}
(w * f)(x,y) = \sum_{s=-a}^{a} \sum_{t=-b}^{b} w(s,t)f(x-s,y-t) \eqnum{3-35}
\end{equation}
\begin{alertblock}{\footnotesize Interpretación de los Signos}
{\footnotesize Los signos negativos en los argumentos de $f(x-s, y-t)$ alinean las coordenadas de $f$ y $w$ cuando una de las funciones se rota 180°. Esta ecuación implementa el proceso que se conoce como \textbf{filtrado espacial lineal}.}
\end{alertblock}
\textbf{\footnotesize Filtrado espacial lineal y convolución espacial son sinónimos.}
\end{frame}

\begin{frame}{Manejo de Bordes: Padding}
{\footnotesize
Cuando el kernel se superpone con los bordes de la imagen, algunos de sus coeficientes caen fuera del área de la imagen.
\begin{itemize}
\item La suma de productos queda indefinida en estas áreas.
\item \textbf{Solución: Padding (Relleno)}. Consiste en añadir píxeles alrededor de la imagen.
\item \textbf{Zero-padding}: El método más común es rellenar con ceros.
\item Si el kernel es de tamaño $1 \times m$ (en 1D), se necesitan $(m-1)/2$ ceros a cada lado de la función $f$.
\item Para un kernel 2D de $m \times n$, se rellena con un mínimo de $(m-1)/2$ filas de ceros arriba/abajo y $(n-1)/2$ columnas de ceros a izquierda/derecha.
\item El padding asegura que el centro del kernel pueda visitar cada píxel de la imagen original.
\end{itemize}
\begin{exampleblock}{Otras opciones de padding}
Existen otras estrategias de padding además del relleno con ceros (e.g., replicar bordes, padding simétrico), que se discuten en detalle en otras partes del material original.
\end{exampleblock}
}
\end{frame}

\begin{frame}[fragile]{Correlación/Convolución con un Impulso Discreto}
Un \textbf{impulso unitario discreto} es una función que es 1 en una ubicación y 0 en todas las demás.
Para una imagen 2D, un impulso de amplitud $A$ en $(x_0, y_0)$ es:
\begin{equation}
\delta(x-x_0, y-y_0) = \begin{cases} A & \text{si } x=x_0, y=y_0 \\ 0 & \text{en otro caso} \end{cases} \eqnum{3-33}
\end{equation}
(Para un impulso unitario, $A=1$).
\vspace{0.1cm}
\begin{columns}[T]
\begin{column}{0.5\textwidth}
\begin{block}{\footnotesize Correlación con un Impulso}
{\footnotesize Correlacionar un kernel $w$ con un impulso unitario discreto produce una copia de $w$ \textbf{rotada 180°} en la ubicación del impulso.}
\end{block}
\end{column}
\begin{column}{0.5\textwidth}
\begin{alertblock}{\footnotesize Convolución con un Impulso}
{\footnotesize Convolucionar un kernel (o cualquier función) $w$ con un impulso unitario discreto produce una \textbf{copia exacta} de $w$ en la ubicación del impulso. Propiedad fundamental en sistemas lineales.}
\end{alertblock}
\end{column}
\end{columns}
\vspace{0.1cm}
\textit{\tiny Nota: Figuras 3.29 y 3.30 del texto original ilustran estos procesos detalladamente para casos 1D y 2D.}
\end{frame}

\begin{frame}{Ilustración Conceptual (Fig. 3.29 y 3.30)}
{\footnotesize
\begin{block}{Idea Principal de Fig. 3.29 (1D) y Fig. 3.30 (2D)}
Estas figuras muestran secuencias paso a paso de:
\begin{itemize}
\item Una función/imagen $f$ (que es un impulso unitario).
\item Un kernel $w$ (no necesariamente simétrico).
\item El proceso de padding de $f$.
\item \textbf{Correlación}:
\begin{itemize}
\item Deslizamiento del kernel $w$ sobre $f$ (paddeada).
\item Cálculo de la suma de productos en cada desplazamiento.
\item Resultado: versión del kernel $w$ \textbf{rotada 180°}, centrada en la posición original del impulso.
\end{itemize}
\item \textbf{Convolución}:
\begin{itemize}
\item El kernel $w$ se rota 180° \textit{antes} del proceso de deslizamiento.
\item Deslizamiento del kernel $w_{rotado}$ sobre $f$ (paddeada).
\item Cálculo de la suma de productos.
\item Resultado: copia \textbf{exacta} del kernel $w$ original, centrada en la posición del impulso.
\end{itemize}
\end{itemize}
\end{block}
\begin{exampleblock}{Simetría del Kernel}
Si el kernel $w$ es simétrico respecto a su centro (i.e., $w$ es igual a $w$ rotado 180°), entonces la correlación y la convolución producen el mismo resultado.
\end{exampleblock}
}
\end{frame}

\begin{frame}[fragile]{Tamaño del Resultado: "Full" vs "Same"}
{\footnotesize
La cantidad de padding afecta el tamaño de la imagen resultante $g(x,y)$.
\begin{itemize}
\item \textbf{Padding para "Same" output}: Si se paddea la imagen $f$ tal que el centro del kernel visita cada píxel original de $f$ y la imagen resultante $g$ tiene el \textit{mismo tamaño} que $f$. Es el caso más común. Se necesitan $(m-1)/2$ y $(n-1)/2$ elementos de padding.
\item \textbf{Padding para "Full" output}: Si se desea que \textit{cada elemento} del kernel $w$ ($m \times n$) visite cada píxel de la imagen $f$ ($M \times N$).
\begin{itemize}
\item Requiere más padding: $(m-1)$ filas y $(n-1)$ columnas.
\item El tamaño del arreglo resultante $(S_v \times S_h)$ es:
\begin{align}
S_v &= M + m - 1 \eqnum{3-36} \\
S_h &= N + n - 1 \eqnum{3-37}
\end{align}
\end{itemize}
\end{itemize}
\textit{Nota: En Ecs. (3-36) y (3-37) del texto, $M,N$ son dims. de imagen y $m,n$ del kernel.}
}
\end{frame}

\begin{frame}{Consideraciones Algorítmicas}
{\footnotesize
\begin{itemize}
\item Muchos algoritmos de filtrado espacial están basados en la correlación (Ec. 3-34).
\item Para usar un algoritmo de \textbf{correlación} para realizar \textbf{convolución}: se introduce el kernel $w$ rotado 180°.
\item Para usar un algoritmo de \textbf{convolución} (Ec. 3-35) para realizar \textbf{correlación}: se introduce el kernel $w$ rotado 180° (menos común).
\item Es decir, cualquiera de las Ecs. (3-34) o (3-35) puede realizar la función de la otra rotando el kernel.
\item \textbf{Importante}: El orden de las funciones en un algoritmo de correlación sí importa, porque la correlación no es conmutativa ni asociativa (ver Tabla 3.5).
\end{itemize}
}
\end{frame}

\begin{frame}{Propiedades Fundamentales (Tabla 3.5)}
\begin{table}
\centering
\caption{\footnotesize Propiedades fundamentales de la convolución y correlación.}
\begin{tabular}{l|>{\centering\arraybackslash}p{3cm}|>{\centering\arraybackslash}p{3cm}}
\hline
\textbf{\scriptsize Propiedad} & \textbf{\scriptsize Convolución ($*$)} & \textbf{\scriptsize Correlación ($\star$)} \\
\hline \hline
\scriptsize Conmutativa & \scriptsize $f*g = g*f$ & \scriptsize --- \\
\scriptsize Asociativa & \scriptsize $f*(g*h) = (f*g)*h$ & \scriptsize --- \\
\scriptsize Distributiva & \scriptsize $f*(g+h) = (f*g) + (f*h)$ & \scriptsize $f \star (g+h) = (f \star g) + (f \star h)$ \\
\hline
\end{tabular}
\vspace{0.1cm}
\tiny (Un "---" indica que la propiedad no se cumple.)
\end{table}
\vspace{0.1cm}
\begin{alertblock}{\footnotesize Relevancia de las Propiedades de la Convolución}
{\footnotesize La conmutatividad y asociatividad de la convolución son cruciales:
\begin{itemize}
\item Permiten el \textbf{filtrado secuencial} eficiente.
\item Son la base de la \textbf{teoría de sistemas lineales e invariantes al desplazamiento (LSI)}.
\item Permiten la \textbf{separabilidad de kernels} (kernel 2D $\rightarrow$ dos convoluciones 1D).
\end{itemize}
}
\end{alertblock}
\end{frame}

%------------------------------------------------------------
% Sección 4: Kernels de Filtrado y Filtrado Secuencial
%------------------------------------------------------------
\section{Kernels y Filtrado Secuencial}

\begin{frame}[fragile]{Ejemplos de Kernels de Suavizado (Fig. 3.31)}
{\footnotesize
Los kernels de suavizado (filtros paso-bajas) se usan para reducir ruido y difuminar detalles.
\begin{columns}[T]
\begin{column}{0.48\textwidth}
\textbf{(a) Box Kernel (Promediador):}
\begin{center}
\begin{tikzpicture}[scale=0.6, every node/.style={scale=0.6}]
\node[left=0.05cm] {$\frac{1}{9} \times$};
\matrix (box) [matrix of math nodes, nodes={draw, minimum size=0.6cm}, right=0.05cm of current bounding box.west, anchor=west, ampersand replacement=\&] {
1 \& 1 \& 1 \\
1 \& 1 \& 1 \\
1 \& 1 \& 1 \\
};
\end{tikzpicture}
\end{center}
\tiny Reemplaza cada píxel por el promedio de su vecindad $3 \times 3$. Todos los coeficientes son iguales.
\end{column}
\begin{column}{0.04\textwidth} \end{column} % Separador
\begin{column}{0.48\textwidth}
\textbf{(b) Gaussian Kernel (Ponderado):}
\begin{center}
\begin{tikzpicture}[scale=0.6, every node/.style={scale=0.6}]
\node[left=0.05cm] {$\frac{1}{4.8976} \times$};
\matrix (gauss) [matrix of math nodes, nodes={draw, minimum size=0.6cm, inner sep=1pt}, right=0.05cm of current bounding box.west, anchor=west, ampersand replacement=\&] {
0.3679 \& 0.6065 \& 0.3679 \\
0.6065 \& 1.0000 \& 0.6065 \\
0.3679 \& 0.6065 \& 0.3679 \\
};
\end{tikzpicture}
\end{center}
\tiny Los coeficientes siguen una Gaussiana. Dan más peso a píxeles centrales. Factor de norm. es suma de coefs.
\end{column}
\end{columns}
\vspace{0.1cm}
\begin{block}{\tiny Nota sobre Simetría} % Cambiado a tiny para el título del bloque también
\tiny Ambos kernels son simétricos respecto a su centro. Por lo tanto, para estos kernels, la correlación y la convolución producirían el mismo resultado. No se requiere rotación previa para la convolución.
\end{block}
}
\end{frame}

\begin{frame}{Terminología en la Literatura}
{\footnotesize
Es común encontrar una terminología algo imprecisa:
\begin{itemize}
\item \textbf{"Filtro de convolución", "máscara de convolución", "kernel de convolución"} a menudo se usan para denotar un kernel de filtro espacial, sin implicar necesariamente que el kernel se use estrictamente para la operación matemática de convolución (podría usarse en una correlación).
\item \textbf{"Convolucionar un kernel con una imagen"} se usa frecuentemente de manera genérica para referirse al proceso de deslizamiento y suma de productos, sin diferenciar estrictamente entre correlación y convolución.
\end{itemize}
\begin{alertblock}{En este contexto (y en gran parte de la teoría de procesamiento de imágenes):}
Cuando se habla de \textbf{filtrado espacial lineal}, se refiere a la \textbf{convolución} de un kernel con una imagen.
\end{alertblock}
}
\end{frame}

\begin{frame}[fragile]{Filtrado Secuencial}
Una imagen $f$ puede ser filtrada secuencialmente, en $Q$ etapas, usando un kernel diferente ($w_1, w_2, \dots, w_Q$) en cada etapa:
$f \xrightarrow{w_1} g_1 \xrightarrow{w_2} g_2 \dots \xrightarrow{w_Q} g_Q$
\vspace{0.1cm}
Debido a la propiedad \textbf{asociativa} de la convolución, esta operación multi-etapa es equivalente a una única operación de filtrado $w*f$, donde el kernel efectivo $w$ es la convolución de todos los kernels individuales:
\begin{equation}
w = w_1 * w_2 * w_3 * \dots * w_Q \eqnum{3-38}
\end{equation}
\vspace{0.1cm}
\begin{exampleblock}{\footnotesize Importancia}
{\footnotesize Esto permite, por ejemplo, diseñar filtros complejos a partir de filtros más simples, o analizar el efecto combinado de varias operaciones de filtrado. No se podría escribir una ecuación similar para la correlación debido a su falta de asociatividad.}
\end{exampleblock}
\end{frame}

\begin{frame}[fragile]{Tamaño del Kernel en Filtrado Secuencial}
{\footnotesize
Si todos los kernels individuales $w_i$ en un filtrado secuencial de $Q$ etapas son de tamaño $m \times n$, y se asume convolución "full" en cada paso intermedio (es decir, cada valor de un kernel visita cada valor del arreglo resultante del paso anterior), el kernel combinado $w$ (de la Ec. 3-38) será de tamaño $W_v \times W_h$:
\begin{align}
W_v &= Q \times (m-1) + m \eqnum{3-39} \\
W_h &= Q \times (n-1) + n \eqnum{3-40}
\end{align}
\vspace{0.1cm}
\begin{itemize}
\item Estas ecuaciones se derivan de aplicaciones sucesivas de las Ecs. (3-36) y (3-37) para el tamaño de una convolución "full".
\item Ejemplo: Si $Q=2$ kernels de $3 \times 3$ ($m=3, n=3$):
\begin{itemize}
\item $W_v = 2 \times (3-1) + 3 = 2 \times 2 + 3 = 7$
\item $W_h = 2 \times (3-1) + 3 = 2 \times 2 + 3 = 7$
\item El kernel combinado $w = w_1 * w_2$ sería de $7 \times 7$.
\end{itemize}
\end{itemize}
}
\end{frame}

%------------------------------------------------------------
% Sección 5: Resumen y Conclusiones
%------------------------------------------------------------
\section{Resumen y Conclusiones}

\begin{frame}{Puntos Clave del Filtrado Espacial Lineal}
{\footnotesize
\begin{itemize}
\item Modifica píxeles mediante una \textbf{suma de productos} con un \textbf{kernel}.
\item La \textbf{correlación} $(w \star f)(x,y) = \sum \sum w(s,t)f(x+s,y+t)$ es la aplicación directa de esta suma.
\item La \textbf{convolución} $(w * f)(x,y) = \sum \sum w(s,t)f(x-s,y-t)$ implica rotar el kernel 180° (o ajustar índices) y es la base del filtrado lineal.
\item Operar con un \textbf{impulso unitario} revela la diferencia:
\begin{itemize}
\item Correlación $\rightarrow$ kernel rotado.
\item Convolución $\rightarrow$ kernel original.
\end{itemize}
\item El \textbf{padding} es crucial para manejar los bordes de la imagen y controlar el tamaño de la salida.
\item La convolución es \textbf{conmutativa y asociativa}, propiedades vitales para la teoría de sistemas y el filtrado secuencial eficiente.
\item El tamaño del kernel resultante de un filtrado secuencial puede determinarse a partir de los tamaños de los kernels individuales.
\end{itemize}
}
\end{frame}

\begin{frame}{Conclusiones}
\begin{block}{\footnotesize Importancia del Filtrado Espacial}
{\footnotesize El filtrado espacial lineal, fundamentado en la operación de convolución, es una de las herramientas más versátiles y fundamentales en el procesamiento digital de imágenes.
\vspace{0.1cm}
Comprender su mecánica, la diferencia entre correlación y convolución, y las propiedades de estas operaciones es esencial para:
\begin{itemize}
\item Diseñar y aplicar filtros para una amplia gama de tareas (suavizado, realce de bordes, detección de características, etc.).
\item Analizar el comportamiento de sistemas de procesamiento de imágenes.
\item Desarrollar algoritmos eficientes.
\end{itemize}
}
\end{block}
\vspace{0.1cm}
\begin{alertblock}{\footnotesize Próximos Pasos}
{\footnotesize Las secciones y capítulos subsecuentes del material original explorarán tipos específicos de filtros espaciales (lineales y no lineales) y sus aplicaciones, así como el filtrado en el dominio de la frecuencia.}
\end{alertblock}
\end{frame}


%------------------------------------------------------------
% Diapositiva final
%------------------------------------------------------------
\begin{frame}
\centering
\vspace*{\fill}
\Huge{\textbf{¿Preguntas?}}
\vspace{0.5cm}
\Large{¡Gracias por su atención!}
\vspace*{\fill}
\end{frame}

%------------------------------------------------------------
% Fin del documento
%------------------------------------------------------------
\end{document}